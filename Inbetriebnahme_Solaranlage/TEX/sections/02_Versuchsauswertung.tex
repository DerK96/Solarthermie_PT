\section{Versuchsauswertung}

\subsection{Standard-Versuchsanlage} 

\subsubsection{Dimensionierung des MAGs}
Für die Inbetriebnahme der Standard-Druckanlage sollte zunächst die Dimensionierung des Ausdehnungsgefäßes (MAG) mit Hilfe der Gleichung nach Schnauss erfolgen.  Nach folgender Gleichung X gilt [aus Praktikumsskript entnommen]: 

\begin{equation}
\label{eq:eff.VolMAG}
V_u= \frac{(V_t * C_e + V_{vap} + V_r)*(p_{max} + 1)}{(p_{max}-p_{min})}
\end{equation}

\begin{center}
	\begin{small}
		$V_u$: Effektives Volumen des MAGs,
		$V_t$: Flüssigkeitsvolumen des Solarkreises,
		$C_e$: Ausdehnungskoeffizient des Wärmeträgerfluids,
		$V_{vap}$: entstehendes Dampfvolumen im Stagnationsfall,
		$V_r$: Reservevolumen,
		$p_{max}$:maximaler Druck $=p_vs*0,9$,
		$p_{min}$: minimaler Druck,
		$p_{vs}$: Auslösedruck des Sicherheitsventils.
	\end{small}
\end{center}

Das Flüssigkeitsvolumen des Solarkreises setzt sich aus den Volumen des Kollektors, dem Inhalt der Rohrleitung und dem Inhalts der Amaturen und des Wärmeübertragers zusammen. Bei dem Kollektor handelt es sich um einen Flachkollektor des Typ Vaillant auroTHERM classic VFK 140/2D* mit gutmütigem Ausdampfverhalten ohne direkten Verbindungsleitungen und einem Fluidinhalt von \SI{1,35}{\liter}. Für das Rohleitungssystem wurde eine Länge von \SI{30}{\metre} angenommen, bei denen es sich jeweils um \SI{50}{\percent} Kupferrohr des Typs DN15 (\SI{0,1771}{\litre\per\metre}) und \SI{50}{\percent} Edelstahl-WR DN16 (\SI{0,2641}{\litre\per\metre}). Der interne Wärmeübertrager wurde mit \SI{2}{\metre} Edelstahl-WR DN20 mit \SI{0,3941}{\litre\per\metre} angenommen. Die Anlagenhöhe beträgt ca. \SI{8}{\metre} und der Auslösedruck des Sicherheitsventils ist mit \SI{6}{\bar} angegeben. Als Wärmeträgerfluid wurde TYFOCOR LS angenommen. Aus dem Datenblatt wurde einen kubischer Ausdehnungskoeffizient von \SI{41,5}{*10^-5\per\kelvin} und eine Dichte von \SI{1013,5}{\kg\per\litre} für eine Umgebungstemperatur von \SI{25}{\celsius} approximiert. Unter der Angabe, dass der Kollektor keine direkte Verbindungsleitungen besitzt, beträgt das Verdampfungsvolumen näherungsweise dem Kollektorvolumen. Der minimale Anlagendruck beträgt dabei die Summe des statischen Drucks addiert zu \SI{0,5}{\bar} für den gasseitigen Druck und \SI{0,3}{\bar} für die Sicherheit. 

$V_u$ unter den angegebenen Bedingung folgenden Wert an:


\begin{equation}
\label{MAG-Ber.}
\begin{split}
V_u= ((\SI{1,35}{\liter}+\SI{15}{\metre}*\SI{0,1771}{\liter\per\metre}+\SI{15}{\metre}*\SI{0,2641}{\liter\per\metre}+\SI{2}{\metre}*\SI{0,3941}{\liter\per\metre}+ 0,3*\SI{6,6180}{\liter}) * \SI{41,5}{*10^-5\per\kelvin} + \SI{1,35}{\liter} + \SI{5}{\liter}) *\frac{(\SI{5,4}{\bar} + 1)}{(\SI{5,4}{\bar}-\SI{1,58}{\bar})}
\end{split}
\end{equation}


\subsection{Drainback-System}

\subsection{Überprüfen des Wasser-Glykol-Gemischs}
