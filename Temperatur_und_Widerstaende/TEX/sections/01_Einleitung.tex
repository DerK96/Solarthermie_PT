\section{Einleitung}

Für den Versuch Temperaturmessung sollten verschiedene Temperatursensoren (Pt100 2L, Pt1000 2L, KTY 2L, NTC 2L) und ein Thermoelement Typ K an einem Metallblockkalibrator mit einem Pt-100 4L Referenzfühler, der zuvor an einer Wassertripelpunktzelle kalibriert wurde,  gemessen und deren Genauigkeit beurteilt werden. 
Die verschiedenen Temperaturensensoren wurden anhand von Fixpunkt- und Vergleichsmethode charakterisiert. 

Anschließend wurden Widerstandssensoren unterschiedlichen Typs mit einem Mulitmeter untersucht und Anhand ihrer Kennliniendiagramme identifiziert. 

Im dritten Teil wurde sich mit der berührungslosen Temperaturmessung befasst. Es erfolgte die Untersuchung verschiedener Materialien in einem temperierten Wasserbad mithilfe einer Wärmebildkamera. Dabei wurden die Unterschiede der Materialien und ihren Einfluss auf die Messung untersucht und analysiert. 